%% Tipografia / Fontes
%% AVISO: Todas essas fontes são *bastante semelhantes* aos nomes com as quais as descrevo. Entenda: são iguais, só que oficialmente com outro nome.

%% %%%%%%%%%%%%%%%%%%%%%%%%%%%%%%%%%%%%%%%%%%%%%%%%%%%%% %%
%% Comente todas as outras fontes que você não vai usar! %%
%% %%%%%%%%%%%%%%%%%%%%%%%%%%%%%%%%%%%%%%%%%%%%%%%%%%%%% %%

%% Latin Modern (fonte padrão do LaTeX, Computer Modern, mas com suporte a caracteres acentuados)
%% Considerada a mais clássica e bonita
%\usepackage{lmodern}



%% Times
%% Considerada a mais confortável de ler quando impresso
%%\usepackage{mathptmx}

%% Variação da mesma fonte, com minúsculas diferenças entre uma e outra (coisas bastante técnicas como kerning, aliasing e afins) - Essa tem revisões frequentes
%\usepackage{newtxtext} \usepackage{newtxmath}



%% Arial
%% Considerada mais confortável de ler num computador
%% ** Oficialmente recomendada pelo manual de formatação do IFPI **
\usepackage{helvet} \renewcommand{\familydefault}{\sfdefault}



%% Palatino
%% Uma opção mais elegante à Times
%\usepackage{newpxtext}



%% Kepler
%% Variação evoluída da Palatino, com várias pequenas diferenças e refinamentos
%\usepackage{kpfonts}



%% Libertine
%% Uma fonte estilo Serif comum no Linux
%\usepackage{libertine} %\usepackage[libertine]{newtxmath}